\section{Introduction}\label{sec:intro}
The very first day of this lab exercise was spent repeating/getting to know the developing- and control environment, Simulink and QuaRC, before moving on to making an optimal control system to calculate the optimal trajectory of a fixed helicopter. In this report the reader is first shown the physical model that forms the basis of the control problem. Section \ref{sec:10.2} explains how an optimal input sequence is calculated and implemented without feedback, to move the helicopter from 0 to 180 degrees. In this case only pitch and travel is controlled. Feedback is then implemented in Section \ref{sec:10.3}, and in Section \ref{sec:10.4} an extra dimension is introduced - elevation. Each section is ended off with a section of results and discussion. The main goal of the report is to give a glance into how optimal control can be implemented to ensure optimal control of physical systems. In practice, this could mean saving fuel, wear or getting just the right set points in a chemical reactor - which again means helping the environment and lowering costs. 











